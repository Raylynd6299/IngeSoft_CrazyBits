\chapter{Stakeholders}
\begin{itemize}
    \item Analista del proyecto: Es el encargo de analizar las necesidades del cliente y traducirlas para que los desarrolladores entiendan que es lo que requiere el cliente y realizarlo.
    \item Equipo de desarrollo: Conformado por programadores, tester’s, diseñadores, documentadores, asegurador de calidad, entre otros.
    \item Jefe de proyecto: El encargado de la comunicación directa de quien contrata, el es quien debe establecer y actualizar los requerimientos del sistema.
    \item Ciudadano: Puede tratarse de algún conductor de vehículo, pasajero o cualquier persona en general que viva o transite en la CDMX, ya sea que le afecte de manera directa o indirecta la existencia de algún bache.
    \item Brigadistas: Personas encargadas de detectar los baches que hay en la Ciudad de México y de reportarlos.
    \item Analistas de Reportes: Es el encargado de verificar si el desperfecto de la calle reportado por el ciudadano cuenta con las características de un bache. 
    \item Permisionarios: Son las empresas externas al gobierno de la Ciudad de México que son contratadas para reparar los baches de la CDMX.
    \item Gestor de reportes: Es el encargado de conducir los reportes atreves de las diversas etapas.
    \item Secretaria de obras y servicios de la CDMX: Dependencia que tiene a su cargo las materias relativas a obras públicas y servicios urbanos. Principal responsable de políticas de construcción y remodelación urbana. Ubicado en: Plaza de la Constitución 1, Centro  histórico de la Ciudad de México, Centro, Cuauhtémoc, 06000 Ciudad de México, CDMX.
    \item Aseguradoras: Se encargan de absorber los gastos derivados de los siniestros reportados.
    \item Proveedores de servicios de la empresa:Son quienes proveerán de servicios a la empresa que va a desarrollar el sistema, entre ellos están los proveedores de: luz, internet, agua, equipos y licencias de Software.
\end{itemize}


