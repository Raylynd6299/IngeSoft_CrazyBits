\chapter{Objetivos específicos}
Los objetivos para realizar el proyecto, derivados de las necesidad de los Stakeholders primarios son:

\begin{enumerate}
     \item Ofrecer a la Secretaría de obras y servicios de la CDMX, permisionarios y aseguradora, una herramienta que les permita recibir reportes de los baches de la CDMX de manera ordenada y con la información precisa.
    \item Desarrollar y realizar la entrega de los reportes de baches y daños de los ciudadanos de la CDMX a la Secretaría de Obras, permisionarios y aseguradora para su pronta atención y reparación.
    \item Brindar una herramienta de fácil adquisición para que los ciudadanos de la CDMX puedan realizar sus reportes de baches.
    \item Adquirir el equipo y licencias necesarias para realizar el proyecto adecuadamente.
    \item Contratar los servicios necesarios (agua, luz e internet) para los diferentes trabajadores y realizar los pagos correspondientes para mantener dichos servicios.
    \item Ofrecer a los permisionarios de la CDMX una herramienta que les permita conocer la ubicación exacta los baches que van a reparar.
    
    \item Facilitar el seguimiento del proceso de reparación baches a los gestores de reportes para asegurar el éxito de dicho proceso.
    \item Desarrollar un visualizador de estados de los baches reportados para que los usuarios puedan consultar los estados de sus reportes.
    \item Desarrollar un visualizador de los históricos de baches detectados y reportados por vialidad, alcaldía, mes, aseguradora y permisionario.
\end{enumerate}

\chapter{Objetivo General}
Realizar un Servicio Web para ayudar a los ciudadanos de la Ciudad de México a reportar los baches que hay en ella y los posibles daños que hayan ocasionado. Al mismo tiempo, brindar a  la Secretaría de Obras y Servicios una herramienta para sus analistas, gestores y permisionario, que les ayude a agilizar sus procesos operativos y administrativos, como a su vez brindarles la localización de los baches y la información para la indemnización de lo ciudadanos afectados y a su vez facilitar la comunicación entre los dependientes de la Secretaria con las personas que han realizado los reportes.

