\chapter{Objetivos específicos}
Los objetivos para realizar el proyecto, derivados de las necesidad de los Stakeholders primarios son:

\begin{itemize}
    \item Realizar una herramienta óptima, seguro y fiable para agilizar los procesos administrativos de la Agencia de Gestión Urbana.
    \item Garantizar que los reportes de baches y daños, de los ciudadanos de la CDMX, llegarán a las autoridades correspondientes para su pronta atención y reparación.
    \item Brindar una herramienta, de fácil adquisición, para que los ciudadanos de la CDMX puedan realizar sus reportes.
    \item Administrar de manera eficiente los recursos económicos, materiales, y humanos para poder concluir de manera puntual y eficaz.
    \item Adquirir el equipo y licencias necesarias para realizar el proyecto adecuadamente.
    \item Contratar los servicios necesarios (agua, luz e internet) para los diferentes trabajadores y realizar los pagos correspondientes para mantener dichos servicios.
    \item Ofrecer a los ajustadores de la CDMX herramientas que les permita conocer la ubicación exacta los baches que van a revisar y notificar si los supuestos baches deben o no repararse.
    \item Ofrecer, a los Analistas de reportes, una herramienta que les permita recibir reportes de los baches de la CDMX de manera ordenada y con la información precisa.
    \item Facilitar el seguimiento del proceso de reparación baches a los Analistas de reportes para asegurar el éxito de dicho proceso.
    \item Desarrollar un visualizador de estados de los baches reportados para que los usuarios puedan consultar los estados de sus reportes.
    
\end{itemize}
