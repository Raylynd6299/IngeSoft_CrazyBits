\chapter{Lista de necesidades}
Las necesidades de cada Stakeholder serán mostradas a continuación. 

\section {Analista del proyecto}
    
\begin{longtable}{|m{1.5cm}|m{3cm}|m{5cm}|m{2cm}| m{2cm}|}
        \rowcolor[HTML]{3531FF} 
        {\color[HTML]{FFFFFF} Id} &{\color[HTML]{FFFFFF}Nombre} & {\color[HTML]{FFFFFF} Descripción} & {\color[HTML]{FFFFFF}Prioridad} & {\color[HTML]{FFFFFF}Alcance}\\
        \hline
        \endfirsthead
        % aquí añadimos el encabezado del resto de hojas.
        \hline
        \rowcolor[HTML]{3531FF} 
        {\color[HTML]{FFFFFF} Id} &{\color[HTML]{FFFFFF}Nombre} & {\color[HTML]{FFFFFF} Descripción} & {\color[HTML]{FFFFFF}Prioridad} & {\color[HTML]{FFFFFF}Alcance}\\
        \hline 
        \endhead
        % aquí añadimos el fondo de todas las hojas, excepto de la última.
        \multicolumn{5}{c}{Sigue en la página siguiente.}
        \endfoot
        % aquí añadimos el fondo de la última hoja.
        \endlastfoot
        NAP01 & Equipo de cómputo & Necesita contar con un equipo de cómputo para realizar la comunicación con el cliente y el equipo de desarrollo & Media & No \\ \hline
        NAP02 & Conexión a internet & Necesita contar con un equipo de computo para realizar la comunicación con el cliente y el equipo de desarrollo & Alto & No \\ \hline
    \caption{Necesidades de los analistas}
    \label{tab:NAPs}
\end{longtable}

\section {Equipo de desarrollo}

\begin{longtable}{|m{1.5cm}|m{3cm}|m{5cm}|m{2cm}| m{2cm}|}
        \rowcolor[HTML]{3531FF} 
        {\color[HTML]{FFFFFF} Id} &{\color[HTML]{FFFFFF}Nombre} & {\color[HTML]{FFFFFF} Descripción} & {\color[HTML]{FFFFFF}Prioridad} & {\color[HTML]{FFFFFF}Alcance}\\
        \hline
        \endfirsthead
        % aquí añadimos el encabezado del resto de hojas.
        \hline
        \rowcolor[HTML]{3531FF} 
        {\color[HTML]{FFFFFF} Id} &{\color[HTML]{FFFFFF}Nombre} & {\color[HTML]{FFFFFF} Descripción} & {\color[HTML]{FFFFFF}Prioridad} & {\color[HTML]{FFFFFF}Alcance}\\
        \hline 
        \endhead
        % aquí añadimos el fondo de todas las hojas, excepto de la última.
        \multicolumn{5}{c}{Siguiente elemento en la página siguiente.}
        \endfoot
        % aquí añadimos el fondo de la última hoja.
        \endlastfoot
        
        NED01 & Equipo de cómputo & Necesita contar con un equipo de computo para realizar el desarrollo del sistema & Media & Si \\ \hline
        NED02 & Conexión a internet & Contar con acceso a internet. & Alto & Si \\ \hline
        NED03 & Licencias de Software & Contar con las licencias de software necesarias para realizar el sistema, como ejemplo, licencia de uso profesional de Google Maps. & Media & No \\ \hline
        NED04 & Servicios & Contar con servicios de luz, agua, etc.. & Media & No \\ \hline
    \caption{Necesidades del equipo de desarrollo}
    \label{tab:NEDs}
\end{longtable}


\section {Jefe de proyecto}
\begin{longtable}{|m{1.5cm}|m{3cm}|m{5cm}|m{2cm}| m{2cm}|}\rowcolor[HTML]{3531FF} 
        {\color[HTML]{FFFFFF} Id} &{\color[HTML]{FFFFFF}Nombre} & {\color[HTML]{FFFFFF} Descripción} & {\color[HTML]{FFFFFF}Prioridad} & {\color[HTML]{FFFFFF}Alcance}\\
        \hline
        \endfirsthead
        % aquí añadimos el encabezado del resto de hojas.
        \hline
        Id & Nombre & Descripción & Prioridad & Alcance  \\
        \hline 
        \endhead
        % aquí añadimos el fondo de todas las hojas, excepto de la última.
        \multicolumn{5}{c}{Sigue en la página siguiente.}
        \endfoot
        % aquí añadimos el fondo de la última hoja.
        \endlastfoot
        
        NJP01 & Equipo de cómputo & Necesita contar con un equipo de computo para realizar la comunicación con el analista y el equipo de desarrollo del sistema & Media & No \\
        \hline
        NJP02 & Conexión a internet & Contar con acceso a internet. & Alto & No \\ \hline
    \caption{Necesidades del jefe de proyecto}
    \label{tab:NJPs}
\end{longtable}

\section {Ciudadano} 
\begin{longtable}{|m{1.5cm}|m{3cm}|m{5cm}|m{2cm}| m{2cm}|}\rowcolor[HTML]{3531FF} 
        {\color[HTML]{FFFFFF} Id} &{\color[HTML]{FFFFFF}Nombre} & {\color[HTML]{FFFFFF} Descripción} & {\color[HTML]{FFFFFF}Prioridad} & {\color[HTML]{FFFFFF}Alcance}\\
        \hline
        \endfirsthead
        % aquí añadimos el encabezado del resto de hojas.
        \hline
        {\color[HTML]{FFFFFF} Id} &{\color[HTML]{FFFFFF}Nombre} & {\color[HTML]{FFFFFF} Descripción} & {\color[HTML]{FFFFFF}Prioridad} & {\color[HTML]{FFFFFF}Alcance}\\
        \hline 
        \endhead
        % aquí añadimos el fondo de todas las hojas, excepto de la última.
        \multicolumn{5}{c}{Sigue en la página siguiente.}
        \endfoot
        % aquí añadimos el fondo de la última hoja.
        \endlastfoot
        NCI01 &  Daños & Reportar cuando un bache a causado algún daño a su vehículo en la CDMX. & Alta & Sí \\ \hline
        NCI02 & Compensación & Recibir una compensación cuando ha sufrido algún daños ocasionado por algún bache. & Media & No \\\hline
        NCI03 &  Herramienta & Contar con alguna herramienta que le permita realizar sus reportes y pedir indemnizaciones de manera fácil, rápida y segura. & Alta & Sí \\
      \hline
    \caption{Necesidades del Ciudadano}
    \label{tab:NCIs}
\end{longtable}

\vspace{3cm}

\section {Brigadistas} 
\begin{longtable}{|m{1.5cm}|m{3cm}|m{5cm}|m{2cm}| m{2cm}|}
        \rowcolor[HTML]{3531FF} 
        {\color[HTML]{FFFFFF} Id} &{\color[HTML]{FFFFFF}Nombre} & {\color[HTML]{FFFFFF} Descripción} & {\color[HTML]{FFFFFF}Prioridad} & {\color[HTML]{FFFFFF}Alcance}  \\
        \hline
        \endfirsthead
        % aquí añadimos el encabezado del resto de hojas.
        \hline
        Id & Nombre & Descripción & Prioridad & Alcance  \\
        \hline 
        \endhead
        % aquí añadimos el fondo de todas las hojas, excepto de la última.
        \multicolumn{5}{c}{Sigue en la página siguiente.}
        \endfoot
        % aquí añadimos el fondo de la última hoja.
        \endlastfoot
        NB01 & Herramienta & Contar con una herramienta que le permita identificar el bache reportado . & Alta  & No \\ \hline
        NB02 & Equipo de computo & Contar con equipo de computo con el cual generar un reporte de el bache reportado . & Media  & No \\ \hline
        NB03 & Conexión a internet  & Contar con conexión a internet con el cual poder establecer conexión al portal generador de reportes . & Media  & No \\ \hline
        NB04 & Transporte  & Contar con un medio de transporte para llegar al la ubicación en el cual fue reportado el bache . & Alta  & No \\ \hline
        NB04 & Viáticos  & Pago que recibe el brigadista para los gastos de transporte que puedan surgir en el trayecto . & Media  & No \\ \hline
    \caption{Necesidades de los Brigadistas}
    \label{tab:NBRs}
\end{longtable}

\section {Analistas de Reportes}
\begin{longtable}{|m{1.5cm}|m{3cm}|m{5cm}|m{2cm}| m{2cm}|}
        \rowcolor[HTML]{3531FF} 
        {\color[HTML]{FFFFFF} Id} &{\color[HTML]{FFFFFF}Nombre} & {\color[HTML]{FFFFFF} Descripción} & {\color[HTML]{FFFFFF}Prioridad} & {\color[HTML]{FFFFFF}Alcance}  \\
        \hline
        \endfirsthead
        % aquí añadimos el encabezado del resto de hojas.
        \hline
        Id & Nombre & Descripción & Prioridad & Alcance  \\
        \hline 
        \endhead
        % aquí añadimos el fondo de todas las hojas, excepto de la última.
        \multicolumn{5}{c}{Sigue en la página siguiente.}
        \endfoot
        % aquí añadimos el fondo de la última hoja.
        \endlastfoot
        NAR01 & Herramienta & Contar con una herramienta que le permita recibir los reportes de baches de la CDMX de manera organizada, clara y específica. & Alta  & Si \\ \hline
        
        NAR02 & Seguimiento & Poder dar seguimiento a la reparación de baches. & Alta  & Si \\ \hline
        
        NAR03 & Reportes & Tener todos los reportes de baches que se han hecho, así como el estado de dichos reportes. & Alta  & Si \\ \hline
        
        NAR04 & Solicitudes & Solicitar revisiones de baches a los ajustadores y reparación de los mismos a las empresas correspondientes & Alta  & Si \\ \hline
        
    \caption{Necesidades de los Analistas de reporte}
    \label{tab:NARs}
\end{longtable}

\vspace{3cm}

\section {Permisionarios}
\begin{longtable}{|m{1.5cm}|m{3cm}|m{5cm}|m{2cm}| m{2cm}|}
        \rowcolor[HTML]{3531FF} 
        {\color[HTML]{FFFFFF} Id} &{\color[HTML]{FFFFFF}Nombre} & {\color[HTML]{FFFFFF} Descripción} & {\color[HTML]{FFFFFF}Prioridad} & {\color[HTML]{FFFFFF}Alcance}  \\
        \hline
        \endfirsthead
        % aquí añadimos el encabezado del resto de hojas.
        \hline
        \rowcolor[HTML]{3531FF} 
        {\color[HTML]{FFFFFF} Id} &{\color[HTML]{FFFFFF}Nombre} & {\color[HTML]{FFFFFF} Descripción} & {\color[HTML]{FFFFFF}Prioridad} & {\color[HTML]{FFFFFF}Alcance}  \\
        \hline 
        \endhead
        % aquí añadimos el fondo de todas las hojas, excepto de la última.
        \multicolumn{5}{c}{Sigue en la página siguiente.}
        \endfoot
        % aquí añadimos el fondo de la última hoja.
        \endlastfoot
        NPER01 & Ubicaciones & Conocer la ubicación exacta de los baches que van a reparar. & Alta  & Si \\ \hline
        
        NPER02 & Generar factura & Notificar el precio de la reparación del bache a reparar & Alta  & Si \\ \hline
        
        NPER03 & Notificar finalización & Notificar cuando un bache ya está reparado & Alta  & Si \\ \hline
        
        NPER04 & Dispositivos para visualizar las peticiones & Indica la necesidad de los contratistas por tener un dispositivo para visualizar los reportes que lleguen a necesitar ser reparados & Alta & no \\ \hline
        
    \caption{Necesidades de los Permisionarios}
    \label{tab:PERs}
\end{longtable}

\section {Gestor de reportes} 
\begin{longtable}{|m{1.5cm}|m{3cm}|m{5cm}|m{2cm}| m{2cm}|}\rowcolor[HTML]{3531FF} 
        {\color[HTML]{FFFFFF} Id} &{\color[HTML]{FFFFFF}Nombre} & {\color[HTML]{FFFFFF} Descripción} & {\color[HTML]{FFFFFF}Prioridad} & {\color[HTML]{FFFFFF}Alcance}\\
        \hline
        \endfirsthead
        % aquí añadimos el encabezado del resto de hojas.
        \hline
        Id & Nombre & Descripción & Prioridad & Alcance  \\
        \hline 
        \endhead
        % aquí añadimos el fondo de todas las hojas, excepto de la última.
        \multicolumn{5}{c}{Sigue en la página siguiente.}
        \endfoot
        % aquí añadimos el fondo de la última hoja.
        \endlastfoot
        
        NGR01 & Equipo de computo & Contar con equipo de computo con el cual generar un reporte de el bache reportado . & Media  & No \\
        \hline
        NGR02 & Herramienta & Contar con una herramienta que le permita modificar el estado de los reportes de baches de la CDMX. & Alta  & Si \\ 
        \hline
        NGR03 & Reportes & Tener todos los reportes de baches que se han hecho. & Alta  & Si \\ 
        \hline
        NGR04 & Informes finales & Contar con los informes de los baches reparadas por mes; así como el promedio de tiempo que tomo repararlos, y los costos que tuvieron. & Media.  & Sí \\ 
        \hline
    \caption{Necesidades del gestor de reportes}
    \label{tab:NGRs}
\end{longtable}

%\vspace{2cm}
\section {Secretaria de obras y servicios de la CDMX}

\begin{longtable}{|m{1.5cm}|m{3cm}|m{5cm}|m{2cm}| m{2cm}|}\rowcolor[HTML]{3531FF} 
        {\color[HTML]{FFFFFF} Id} &{\color[HTML]{FFFFFF}Nombre} & {\color[HTML]{FFFFFF} Descripción} & {\color[HTML]{FFFFFF}Prioridad} & {\color[HTML]{FFFFFF}Alcance}\\
        \hline
        \endfirsthead
        % aquí añadimos el encabezado del resto de hojas.
        \hline
        \rowcolor[HTML]{3531FF} 
        {\color[HTML]{FFFFFF} Id} &{\color[HTML]{FFFFFF}Nombre} & {\color[HTML]{FFFFFF} Descripción} & {\color[HTML]{FFFFFF}Prioridad} & {\color[HTML]{FFFFFF}Alcance}\\
        \hline 
        \endhead
        % aquí añadimos el fondo de todas las hojas, excepto de la última.
        \multicolumn{5}{c}{Sigue en la página siguiente.}
        \endfoot
        % aquí añadimos el fondo de la última hoja.
        \endlastfoot
        NSDOS01 & Herramienta de reportes & Contar con una herramienta óptima para la recepción y seguimiento de los reportes de los baches en la CDMX. & Alta & Si \\
        \hline
        
        NSDOS02 & Procesos & Agilizar los procesos de reparación de baches. & Alta & No \\
        \hline
        
        NSDOS03 & Personal & Contar con personal capacitado para dar seguimiento a los reportes. & Media & No\\ 
        \hline
        
        NSDOS04 & Recibir reportes & Recibir los reportes de los baches existentes en la CDMX, así como de los posibles daños que han causado.  & Alta & Si\\
        \hline
        
        NSDOS05 & Datos de aseguradoras & Conocer los datos de las aseguradoras (Periodo de contrato, datos fiscales, datos de contacto, usuarios). & Alta & Sí \\
        \hline
        
        NSDOS06 & Notificar arreglo & Notificar el arreglo de los baches, mediante reportes, a las autoridades correspondientes. & Alta & Si\\
        \hline
        
        NSDOS07 & Datos de permisionarios & Conocer los datos de las aseguradoras (Datos fiscales, de contacto, usuarios, periodo de contrato y alcaldías que tiene a su cargo). & Alta & Sí \\
        \hline
        
        
    \caption{Necesidades de la secretaría de obras y servicios de la CDMX}
    \label{tab:NPDS}
\end{longtable}


\section {Aseguradoras}
\begin{longtable}{|m{1.5cm}|m{3cm}|m{5cm}|m{2cm}| m{2cm}|}\rowcolor[HTML]{3531FF} 
        {\color[HTML]{FFFFFF} Id} &{\color[HTML]{FFFFFF}Nombre} & {\color[HTML]{FFFFFF} Descripción} & {\color[HTML]{FFFFFF}Prioridad} & {\color[HTML]{FFFFFF}Alcance}\\
        \hline
        \endfirsthead
        % aquí añadimos el encabezado del resto de hojas.
        \hline
        Id & Nombre & Descripción & Prioridad & Alcance  \\
        \hline 
        \endhead
        % aquí añadimos el fondo de todas las hojas, excepto de la última.
        \multicolumn{5}{c}{Sigue en la página siguiente.}
        \endfoot
        % aquí añadimos el fondo de la última hoja.
        \endlastfoot
        
        NAE01 & Daños & Conocer los reportes de los baches que han ocasionado daños.& Alta  & Sí \\ \hline
        
        NAE02 & Pagar & Realizar los pagos a las personas afectadas. & Media & No \\\hline
        
        NAE03 & Equipo & Contar con un dispositivo para visualizar los reportes que lleguen a necesitar ser indemnizados. & Alta & No
        \hline
    \caption{Necesidades de las aseguradoras}
    \label{tab:NAEs}
\end{longtable}

\section {Proveedores de servicios de la empresa}
\begin{longtable}{|m{1.5cm}|m{3cm}|m{5cm}|m{2cm}| m{2cm}|}\rowcolor[HTML]{3531FF} 
        {\color[HTML]{FFFFFF} Id} &{\color[HTML]{FFFFFF}Nombre} & {\color[HTML]{FFFFFF} Descripción} & {\color[HTML]{FFFFFF}Prioridad} & {\color[HTML]{FFFFFF}Alcance}\\
        \hline
        \endfirsthead
        % aquí añadimos el encabezado del resto de hojas.
        \hline
        Id & Nombre & Descripción & Prioridad & Alcance  \\
        \hline 
        \endhead
        % aquí añadimos el fondo de todas las hojas, excepto de la última.
        \multicolumn{5}{c}{Sigue en la página siguiente.}
        \endfoot
        % aquí añadimos el fondo de la última hoja.
        \endlastfoot
        
        NPDS01 & Pagos & Recibir los pagos puntuales de los servicios prestados. & Alta & Sí \\
        \hline
    \caption{Necesidades de los proveedores de servicio}
    \label{tab:NPSS}
\end{longtable}